% o \label{codigo} serve para podermos fazer referencias para algo numerado, 
% como capitulos, tabelas, figuras, etc. 
% Quando colocamos o comando \ref{codigo}. o compilador troca o \ref{codigo}
% pelo numero atribuido ao \label{}
% ex. \label{tabelaLegal}
%   A tabela \ref{tabelaLegal} mostra que...
% vai ser substituido por
%   A tabela 2 mostra que

\chapter{Fundamentação Teórica}\label{cap-fundamentacao}

Os conceitos teóricos explorados no planejamento e implementação deste projeto advém de diversas áreas do conhecimento - técnicas, antropológicas, pedagógicas, dentre outras - que foram empregadas conjuntamente para a obtenção do resultado desejado. Em termos gerais, pode-se dividir as disciplinas exploradas dentre três categorias principais:

\begin{itemize}[label={--},noitemsep,topsep=0pt,leftmargin=4mm]
	\item Conceituação pedagógica
	\item Exploração tecnológica
	\item \textit{Game design}
\end{itemize}
 
% ---
\section{Conceituação Pedagógica}\label{sec-fund-conceituacao-pedagogica}
% ---

Uma vez que a preocupação primária do trabalho é a elaboração de um produto que possa auxiliar educadores no ensino de \[...\]

\TODO{O capítulo todo}