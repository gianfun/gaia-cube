% o \label{codigo} serve para podermos fazer referencias para algo numerado, 
% como capitulos, tabelas, figuras, etc. 
% Quando colocamos o comando \ref{codigo}. o compilador troca o \ref{codigo}
% pelo numero atribuido ao \label{}
% ex. \label{tabelaLegal}
%   A tabela \ref{tabelaLegal} mostra que...
% vai ser substituido por
%   A tabela 2 mostra que

\chapter{Teoria}\label{cap-teoria}

O desenvolvimento de um jogo educacional, que funcione como uma ferramenta de auxílio ao aprendizado em sala de aula, deve encompassar a consideração de diversos aspectos pedagógicos, sociais e de *game design*.

Em primeiro lugar, é necessário o estabelecimento de um vocabulário comum e bem definido, delineando conceitos relevantes que possam ser empregado ao longo da formulação e desenvolvimento do projeto.

% ---
\section{Jogos Digitais}\label{sec-jogosdigitais}
% ---

Para \cite{correia:2009:digital_games_spore}, jogos digitais são aqueles aonde exista interação humano-computador, com o uso da tecnologia, e desenvolvidos para serem jogados em computadores, consoles ou outros dispositivos tecnológicos.

Tais jogos podem ser utilizados para auxiliar a assimilação de informações, permitindo o aprendizado através de estratégias de ensino diferentes daquelas exploradas em sala de aula, conforme demonstrado por \cite{fernandes:2012:digital_education}. Isto ocorre, em parte, por incentivar o aprendizado através da prática de atividades.

% ---
\section{Mecânicas, dinâmicas e estética}\label{sec-mecanica-dinamica-estetica}
% ---

Segundo \cite{hunicke:2004}, um jogo, digital ou não, pode ser analisado através de sua decomposição entre três componenentes principais: mecânicas, dinâmicas e estética.

Mecânicas são entendidas como as regras efetivas do jogo, compondo as liberdades concedidas e restrições impostas ao jogador, bem como seus objetivos, condições de vitória e derrota. As mecânicas são a porção do jogo sobre a qual o desenvolvedor tem controle direto, e influenciam a percepção das demais porções por parte do jogador.

Chama-se de dinâmicas os comportamentos emergentes demonstrados por jogadores em função das mecânicas. Encompassam desde comportamentos simples, como a reação de um jogador a obstáculos do cenário, até a sofisticadas estratégias a respeito de quando cooperar ou competir em jogos multi-jogador.

Estéticas definem a experiência essencial que o jogador tem ao longo de uma sessão de jogo. Diz respeito às respostas emocionais, ao envolvimento e à imersão do jogador no mundo virtual criado pelo desenvolvedor, e é útil para categorizar o jogo quanto às sensações que ele é capaz de invocar.


% ---
\section{Realidade Virtual}\label{sec-realidadevirtual}
% ---

A Realidade Virtual é, de acordo com \cite{kirner:2007:RV_e_RA}, uma interface de usuário avançada, que permite que o usuário veja, mova e interaja, em tempo real, com um ambiente tridimensional, através de dispositivos especiais. Em geral, a visão tende a ser priorizada em interfaces de realidade virtual, mas a audição e o toque são também partes muito importantes da experiência do usuário.
Para \cite{kirner:2011:evolucao_RV}, Um mundo virtual visto através de uma tela é considerado não-imersiva, enquanto a percepção deste mundo virtual através de salas com multiprojeção ou por meio de capacetes é considerado imersivo. Desta forma, a realidade virtual:

- Considera a entrada e saída feita com equipamentos capazes de manipular informação multisensorial em tempo real;

- Prioriza interação em tempo real sobre a qualidade das informações;

- Requer processamento gráfico, háptico e sonoro em alta escala;

- Leva em consideração que ações ocorrem em um espaço tridimensional;

- Considera que equipamentos especiais para a interação multisensorial sejam utilizados;

- Entende que o usuário necessita de um período de adaptação para entender a representação virtual do mundo.

% ---
\section{Competências e Habilidades}\label{sec-competenciashabilidades}
% ---

	O Ministerio da Educação (MEC) define um conjunto de parâmetros que devem estar presentes no curriculo de escolas em todo o Brasil. Estes são os Parâmetros Curriculares Nacionais (PCN), e abordam várias diferentes áreas importantes. 
As competências e habilidades de raciocínio lógico consideradas para análise consistem de um subconjunto dos PCN, em especial, daqueles relativos ao raciocínio lógico, e são apresentados na Tabela \ref{tabela:PCN}, retirado de \cite{Tabuti:2015:tabela_habilidades}.

\renewcommand{\arraystretch}{1.2}
\begin{table}[h] \scriptsize
	\centering
	\begin{tabular} {| >{\centering\arraybackslash} m{2cm} | >{\centering\arraybackslash} m{1.5cm} | m{7cm} |}
		\hline
		Competência & Habilidade & \multicolumn{1}{>{\centering\arraybackslash}p{7cm} |}{Descrição da habilidade} \\
		\hline
		\multirow{1}[12]{*}{Análise} & H1 & Habilidade em resolver um problema complexo dividindo-o em subproblemas mais simples, com soluções mais imediatas \\ \cline{2-3}
		& H2 & Habilidade da recomposição dos resultados gerados para a solução do problema mais complexo \\
		\hline
		\multirow{1}[40]{*}{Síntese} & H3 & Habilidade em juntar informações e dados de um problema que sejam de diferentes naturezas \\ \cline{2-3}
		& H4 & Habilidade em avaliar a deficiência dessas informações para determinar a solução\\ \cline{2-3}
		& H5 & Habilidade em descobrir a falta de outras informações necessárias para a resolução do problema  \\ \cline{2-3}
		& H6 & Habilidade em priorizar essas informações para o desenvolvimento até atingir a solução \\ \cline{2-3}
		& H7 & Habilidade em ordenar essas informações de forma a obter uma sequência de desenvolvimento até atingir a solução \\
		\hline
		\multirow{1}[28]{*}{Inferência} & H8 & Habilidade em descobrir padrões em um conjunto de informações \\ \cline{2-3}
		& H9 & Habilidade em aplicar novamente determinada informação de forma a agregar novas informações a estrutura de padrões existente \\ \cline{2-3}
		& H10 & Habilidade em aplicar novamente determinada informação de forma a agregar novas informações num conjunto que preserve a estrutura de padrões existente \\ 
		\hline
	\end{tabular}
	\caption[Competencias e habilidades]{Competências e habilidades relacionadas ao raciocínio lógico}
	\label{tabela:PCN}
\end{table}

As três competências demonstradas na tabela são importantes para a resolução de problemas. O desenvolvimento da competência de análise permite a resolução de problemas através da divisão em problemas menores que tenham soluções mais simples, assim como a recomposição dos resultados gerados para solucionar o problema mais complexo. O desenvolvimento da competência de síntese permite combinar e ordenar diferentes informações obtidas, afim de chegar a uma solução geral. O desenvolvimento da competência de inferência, por sua vez, permite reutilizar informações, padrões e relações aprendidas em situações anteriores para solucionar problemas futuros.

Dentro do escopo deste projeto, prentede-se dar maior enfoque na exploração e desenvolvimento das habilidades H1, H2, H6 e H7.

% ---
\section{Público alvo}\label{sec-publico-alvo}
% ---

A escolha do público alvo do projeto se deu a partir de duas considerações principais: a premissa da criação de um jogo que auxiliasse no aprendizado de habilidades de raciocínio lógico; e a análise de qual a faixa etária que mais se beneficiaria do contato com um jogo nesses moldes.

Pretendeu-se então projetar um jogo voltado para alunos do ensino fundamental, entre 8 a 12 anos, uma faixa etária essencial para a formação das capacidades de raciocínio lógico segundo (------).
%TODO: Terminar PUBLICO ALVO


% ---
\section{Tecnologia}\label{sec-tecnologia}
% ---


% ---
\subsection{Controlador Leap Motion}\label{subsubsec-teo-leap-motion}
% ---

O controlador Leap Motion, da empresa de mesmo nome, começou a ser desenvolvido em 2008, contou com várias rodadas de investimento, e a primeira versão do Software Development Kit foi lançada em maio de 2012. Dois anos depois, foi apresentada a segunda versão do SDK, que melhorou bastante o rastreamento das mãos, e que é atualmente a versão mais utilizada. No início de 2016, a mais nova versão, chamada Orion, foi lançada, desenvolvida especialmente para a realidade virtual.

Para utilizar o controlador, são necessários 3 elementos diferentes: O hardware do controlador em si; um software rodando como um serviço em um computador; e um software 'cliente' que receberá os dados do serviço para utilizá-los.

O controlador nada mais é do que um conjunto de duas câmeras e 3 LEDs, todos infravermelhos. Os LEDs projetam luz infravermelha, que é refletida pelas mãos do usuário e captadas pelas câmeras. Estas imagens são enviadas ao serviço, que utiliza algoritmos de visão computacional para gerar modelos tridimensionais das mãos do usuário. A versão 2 do software aumentou muito a precisão do controlador por começar a trabalhar com um modelo esqueletal da mão, baseado na anatomia humana, ou seja, um modelo composto por 5 dedos, cada um contendo 4 ossos. Este método insere restrições ao modelo, eliminando poses e movimentos que não seriam fisicamente possíveis.

O software cliente, então, pode receber os dados do serviço por duas interfaces: Conversando diretamente com o serviço, utilizando um DLL que acompanha o software, ou acessando o servidor WebSocket que o serviço roda localmente.

A versão mais recente do SDK ainda é nova, e tem focado na otimização dos algoritmos utilizados em várias áreas, tanto no processamento de imagens quanto outros algoritmos de 'suporte', como a detecção da adição ou remoção de um controlador. Visto a sua orientação para a realidade virtual, uma otimização importante foi em relação à mãos ocluídas, que sumiam em versões anteriores, conforme descrito em \cite{leap:2016:changeset}. É possível que comece a focar em coisas mais específicas para a realidade virtual em pouco tempo.

% TODO: Falar que é um dos mais importantes controles?

% ---
\subsection{Google Cardboard}\label{subsec-teo-google-cardboard}
% ---

O \textit{Google Cardboard} é um projeto da empresa Google criada em 2014 para fomentar o desenvolvimento de aplicações de realidade virtual. É composta por um \textit{headset} e de um software compatível. A ideia principal do projeto era conseguir proporcionar a experiência de realidade virtual da forma mais barata o possível. Para alcançar tal objetivo, o \textit{headset} é feito de papelão, de tal forma que o único subcomponente mais caro seja as lentes. Adicionalmente, um celular é utilizado tanto como poder de processamento do \textit{headset} quanto como a tela deste. Visto que uma grande parcela da população ja tem um \textit{smartphone}, estas partes do Cardboard vêm de graça.

Quanto ao software compatível, é necessário que este divida a sua saída de video (que será mostrado na tela do celular) em metades, uma para cada olho. Visto que cada imagem emula a visão de um olho, é necessário que as imagens venham de pontos de origem separados por uma distância equivalente àquela entre as duas pupilas, valor que varia entre 58 mm e 70 mm, de acordo com \cite{dodgson:2004:svariation}. Adicionalmente, devido à distorção causada pelas lentes, é necessário que o software aplique a distorção inversa na imagem, de forma que a imagem após passar pela lente fique correta. É também necessário ler sensores encontrados no dispositivo móvel, como acelerômetros e giroscópios, e disponibilizar estes dados para que se possa implementar o movimento do usuário dentro do jogo ou aplicativo.

O SDK do Google Cardboard, distribuído pela própria Google e disponível online em \cite{google:2016:cardboardSDK}, já faz todas estas transformações necessárias na imagem, além de enviar os dados de movimento, ja processados, de volta ao software que esteja utilizando o SDK.