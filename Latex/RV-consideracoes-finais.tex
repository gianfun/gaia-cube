% o \label{codigo} serve para podermos fazer referencias para algo numerado, 
% como capitulos, tabelas, figuras, etc. 
% Quando colocamos o comando \ref{codigo}. o compilador troca o \ref{codigo}
% pelo numero atribuido ao \label{}
% ex. \label{tabelaLegal}
%   A tabela \ref{tabelaLegal} mostra que...
% vai ser substituido por
%   A tabela 2 mostra que

\chapter{Considerações finais}\label{cap-consideracoes-finais}

Texto texto

% ---
\section{Principais Resultados}\label{sec-conc-resultados}
% ---

Texto numa secao (ex. 2.2)

% ---
\section{Contribuições}\label{sec-conc-contribuicoes}
% ---

As contribuições desse trabalho na nossa formação foram:

\begin{alineas}
	\item Utiliza a versão v2 do \textit{SDK} do \textit{Leap}, que tem precisão pior.
	\item Necessita da versão beta do serviço para o \textit{Android}, de difícil acesso.
\end{alineas}

\begin{table}[htb]
	\IBGEtab{%
		\caption{Um Exemplo de tabela alinhada que pode ser longa
			ou curta, conforme padrão IBGE.}%
		\label{tabela-ibge}
	}{%
		\begin{tabular}{ccc}
			\toprule
			Nome & Nascimento & Documento \\
			\midrule \midrule
			Maria da Silva & 11/11/1111 & 111.111.111-11 \\
			\midrule 
			João Souza & 11/11/2111 & 211.111.111-11 \\
			\midrule 
			Laura Vicuña & 05/04/1891 & 3111.111.111-11 \\
			\bottomrule
		\end{tabular}%
	}{%
		\fonte{Produzido pelos autores.}%
		\nota{Esta é uma nota, que diz que os dados são baseados na
			regressão linear.}%
		\nota[Anotações]{Uma anotação adicional, que pode ser seguida de várias
			outras.}%
	}
\end{table}

 Praticar o gerenciamento de projeto, praticar o trabalho em grupo, e aprender

a lidar com a divisão de tarefas, de forma a manter a coesão do projeto e não

se esquecer da integração entre as partes.

 Criar uma visão do processo envolvido no desenvolvimento de um sistema do

início ao fim.

 Aprender novas tecnologias, linguagens de programação e metodologias de

projeto.

 Aplicar as teorias aprendidas durante o curso e adquirir novos conhecimentos

sempre que necessário.

O que colaborou para o projeto:

 Nossa orientadora, Profa. Lucia Filgueiras, ter conhecimento em Engenharia

de Software e Acessibilidade.

 Nosso co-orientador, Prof. Bruno Albertini, ter conhecimento sobre Hardware

e Sensores.

64

 O grupo já ter trabalhado junto anteriormente, ter um excelente

relacionamento e as habilidades dos membros do grupo serem

complementares.

 Conhecimentos em programação que foram diferenciais: a Ana já sabia

programar em Java e em Swift (devido à iniciação científica que ela realizou

anteriormente) e a Isabela já sabia programar em Python;

O que contribuiu para a engenharia:

 Criação de um arcabouço para lidar com itens e armazenadores, facilitando o

desenvolvimento de Aplicativos, para iOS, acessíveis e integração com uma

rede de sensores para tornar os itens e armazenadores inteligentes.

 Aplicação do conceito de IoT na área de acessibilidade.

% ---
\section{Dificuldades}\label{sec-conc-dificuldades}
% ---

% ---
\section{Trabalhos Futuros}\label{sec-conc-trabalhos-futuros}
% ---