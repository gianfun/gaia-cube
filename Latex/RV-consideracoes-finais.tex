% o \label{codigo} serve para podermos fazer referencias para algo numerado, 
% como capitulos, tabelas, figuras, etc. 
% Quando colocamos o comando \ref{codigo}. o compilador troca o \ref{codigo}
% pelo numero atribuido ao \label{}
% ex. \label{tabelaLegal}
%   A tabela \ref{tabelaLegal} mostra que...
% vai ser substituido por
%   A tabela 2 mostra que

\chapter{Considerações finais}\label{cap-consideracoes-finais}

O objetivo deste trabalho foi criar um jogo efetivamente educacional, capaz de manter o engajamento e interesse do jogador, de forma a ajudar a exercitar habilidades de raciocínio lógico, além de desenvolver uma solução final de baixo custo que possa ser implementada em instituições educacionais.

% ---
\section{Principais resultados}\label{sec-conc-resultados}
% ---

Foi desenvolvido um jogo digital, utilizando realidade virtual 
e o controle através das mãos, que entreteu os jogadores que o 
testaram, mesmo havendo algumas falhas no controlador. 

A solução também tem um custo acessível, se comparado a outras soluções 
de realidade virtual, devido ao uso de componentes de baixo custo e a 
possível reutilização de celulares \textit{smartphones} e \textit{laptops}. 
O jogo roda com latência aceitável, não necessita de um computador com 
componentes caros e de alto desempenho (como uma placa de video dedicada 
topo de linha), e é simples de configurar e executar.

Não foi possível aplicar o questionário criado, e portanto não foi 
possível obter dados qualitativos sobre o efeito do jogo sobre as 
competências e habilidades do raciocínio lógico. Porém, durante a 
sessão de \textit{playtesting} se constatou que os jogadores 
encaravam o jogo como um desafio lógico, comprovando em parte o 
exercício do raciocínio lógico.

% ---
\section{Contribuições}\label{sec-conc-contribuicoes}
% ---

As contribuições desse trabalho na nossa formação foram:

\begin{alineas}
	\item praticar o gerenciamento de projeto e o trabalho em grupo;
	\item criar uma visão do processo envolvido no desenvolvimento de um jogo do início ao fim;
	\item aprender novas tecnologias, como o controlador \textit{Leap Motion}, \textit{WebSockets} e o \textit{Google Cardboard};
	\item aplicar teorias aprendidas durante o curso.
\end{alineas}

Fatores que colaboraram para o projeto foram:

\begin{alineas}
	\item nossa co-orientadora, Profa. Msc. Lucy Mari Tabuti, ter conhecimento e experiência na área de jogos de lógica digitais, jogos educacionais, realidade virtual, e com o controlador \textit{Leap Motion};
	\item os integrantes do grupo ja terem um pouco de experiênca com o \textit{Unity}.
\end{alineas}

% ---
\section{Dificuldades}\label{sec-conc-dificuldades}
% ---

Em relação ao controlador \textit{Leap Motion}, houve dificuldade em detectar
movimentos específicos, tanto por inexperiência (conforme dito na
\autoref{sec-segunda-iteracao-mecanicas-basicas}) quanto por dificuldades 
em expressar os movimentos da forma correta, ao se tratar das relações entre 
os ossos e vetores que definiam a mão do jogador, conforme descrito na
\autoref{subsubsec-teo-gestos}.

Adicionalmente, houve dificuldade de se aplicar os testes em crianças pois
suas mãos, possivelmente pelo seu tamanho menor, não eram detectadas muito 
bem pelo controlador. Conforme descrito na 
\autoref{sec-roteiro-observacoes}, porém, é possível que este problema 
tenha sido devido à iluminação do local, que atrapalha o controlador.

A definição da arquitetura também encontrou obstáculos, visto que a 
primeira escolha de arquitetura teve de ser modificada devido 
a problemas relacionados ao hardware. Também ocorreu problemas devido à falta de documentação de elementos, como da versão \textit{Android} da biblioteca do \textit{Leap Motion} e do software \textit{Riftcat}.

% ---
\section{Trabalhos futuros}\label{sec-conc-trabalhos-futuros}
% ---

Dado que não foi possível aplicar o questionário com pessoas pertencentes ao 
público alvo, um próximo passo seria este. Desta forma, seria possível verificar 
o quão benéfico o jogo é.

Adicionalmente, novas fases com novas mecânicas podem ser implementadas no 
futuro, aumentando o tamanho do jogo, de forma a permitir que jogadores 
passem mais tempo se divertindo, enquanto exercitam suas habilidades de 
raciocínio lógico.

Melhorias no código, em especial no reconhecimento de gestos utilizados 
para ações também são um próximo passo importante, por tornar o jogo 
menos frustrante ao usuário, que não mais precisará repetidamente executar 
um mesmo gesto para tentar ativar uma ação.