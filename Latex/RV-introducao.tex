\chapter{Introdução}

O emprego de jogos digitais como ferramento de auxílio na educação é um conceito estudado há décadas. Tentativas de criar \textit{software} que tome proveito deste tipo de mídia para ensinar determinados tópicos ou fomentar o desenvolvimento de habilidades específicas em crianças existem desde ao menos a década de 1970, com a criação do Consórcio de Computação Educacional de Minnesotta, MECC.

Contudo, diversos fatores impediram, ao longo da história, a implementação de uma estratégia efetiva de desenvolvimeneto e implantação de jogos educativos. Dificuldade em manter o engajamento de alunos, despreparo dos educadores para lidar com novas tecnologias, falhas na criação de um produto que efetivamente ensine as competências corretas, e alto custo de implantação das ferramentas foram grandes obstáculos à adoção desta mídia na sala de aula. \TODO{Fonte?}

...

Recentemente, o tópico do desenvolvimento de \textit{video games} educativos tem voltado a estar em pauta, sobretudo dada a grande revolução tecnológica que tem ocorrido no campo de mídias interativas: a criação de dispositivos de realidade virtual.

...

\section{Objetivos}\label{sec-objetivos}

A partir da realização deste projeto, pretende-se dar os primeiros passos rumo a criação de jogos efetivamente educacionais, capazes de manter o engajamento e interesse de estudantes, ao passo que ensinam sobre tópicos e habilidades relevantes para seu desenvolvimento intelectual e psicológico.

Também pretende-se chegar a uma solução final de baixo custo, que possa ser implementada em larga escala por escolas e instituições educacionais sem impactos orçamentários proibitivos.

\TODO{Faltam objetivos}


\section{Justificativa}\label{sec-justificativas}

\TODO{Escrever. Não precisa ser um section, mas precisa estar claro , assim como a motivação}

\section{Resultados esperados}\label{sec-resultados-esperados}

\TODO{Escrever}

\section{Trabalhos relacionados}\label{sec-trabalhos-relacionados}

\cite{Tabuti:2010:analise} estudam diferentes interações humano-computador, 
focando em como elas ajudam, ou atrapalham, o aprendizado em ambientes 
digitais. O estudo foi feito com base no cubo de Rubik, tanto em sua 
forma física quando em várias formas digitais.

\cite{SBGames:2015:RVAM} desenvolveram um jogo digital em realidade 
virtual, motivados pela baixa performance de alunos em aulas de matemática. 
O foco do jogo é ajudar o jogador aprender e reter conceitos apresentados 
em aula, através do uso de ambientes tridimensionais e com o auxílio de 
personagens e mundo visualmente atraentes. O jogo consiste de um mundo 
aberto e uma personagem controlável. O objetivo do jogo é conseguir o 
maior número de moedas o mais rápido possível, enquanto desviando de 
inimigos. Quando o jogador coleta uma moeda, aparece uma tela com uma questão 
de múltipla escolha. Caso o jogador acerte a pergunta, recebe a moeda 
colecionada, caso contrário é teleportado para o início da fase. O jogo pode 
ser dividido em dois estados: Um estado de 'jogo', quando o jogador está 
caçando as moedas, e um estado de 'aprendizado', quando o jogador está 
respondendo uma questão.

\cite{Alves:2015:VR_Quimica} também desenvolveram um jogo em realidade 
virtual, mas focando no ensino de química. Os gráficos são muito mais 
realistas do que o exemplo anterior, refletindo o público alvo mais 
maduro. O jogador controla uma personagem que é uma antropomorfização 
de um átomo de hidrogênio que tem que explorar o mundo (que é baseado 
na tabela periódica). Cada área diferente do mapa contém elementos 
diferentes, que interagem com o jogador de maneiras únicas. O objetivo 
do jogo é auxiliar o ensino de reações químicas para crianças. Através 
da transformação dos elementos em personagens, os autores tinham como 
objetivo ajudar o jogador a entender a química através de metáforas, 
contrapondo o exemplo anterior, onde o jogo é dividido em estados.
