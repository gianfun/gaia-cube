\chapter{Introdução}

O emprego de jogos digitais como ferramento de auxílio na educação é um conceito estudado há décadas. Tentativas de criar \textit{software} que tome proveito deste tipo de mídia para ensinar determinados tópicos ou fomentar o desenvolvimento de habilidades específicas em crianças existem desde ao menos a década de 1970, com a criação do Consórcio de Computação Educacional de Minnesotta, MECC.

Contudo, diversos fatores impediram, ao longo da história, a implementação de uma estratégia efetiva de desenvolvimeneto e implantação de jogos educativos. Dificuldade em manter o engajamento de alunos, despreparo dos educadores para lidar com novas tecnologias, falhas na criação de um produto que efetivamente ensine as competências corretas, e alto custo de implantação das ferramentas foram grandes obstáculos à adoção desta mídia na sala de aula. \TODO{Fonte?}

...

Recentemente, o tópico do desenvolvimento de \textit{video games} educativos tem voltado a estar em pauta, sobretudo dada a grande revolução tecnológica que tem ocorrido no campo de mídias interativas: a criação de dispositivos de realidade virtual.

...

\section{Objetivos}\label{sec-objetivos}

A partir da realização deste projeto, pretende-se dar os primeiros passos rumo a criação de jogos efetivamente educacionais, capazes de manter o engajamento e interesse de estudantes, ao passo que ensinam sobre tópicos e habilidades relevantes para seu desenvolvimento intelectual e psicológico.

Também pretende-se chegar a uma solução final de baixo custo, que possa ser implementada em larga escala por escolas e instituições educacionais sem impactos orçamentários proibitivos.

\TODO{Faltam objetivos}


\section{Justificativa}\label{sec-justificativas}

\TODO{Escrever. Não precisa ser um section, mas precisa estar claro , assim como a motivação}

\section{Resultados esperados}\label{sec-resultados-esperados}

\TODO{Escrever}