\chapter{Introdução} \label{cap-introducao}

O emprego de jogos digitais como ferramenta de auxílio na 
educação é um conceito estudado há décadas. Tentativas de 
criar \textit{software} que tome proveito deste tipo de 
mídia para ensinar determinados tópicos ou fomentar o 
desenvolvimento de habilidades específicas em crianças 
existem desde ao menos a década de 1970, com a criação do 
Consórcio de Computação Educacional de Minnesotta (MECC), 
cujo objetivo era auxiliar escolas do estado de Minnesota, 
nos EUA, a integrar o uso de computadores ao ensino, de 
acordo com \cite{lussenhop:2016:oregon-trail}.

Além de prover infraestrutura e computadores às escolas 
locais, o MECC também incentivou a criação de inúmeros 
jogos educacionais. Dentre esses destaca-se o 
\textit{Oregon Trail}, possivelmente um dos mais 
clássicos e conhecidos jogos educacionais, que fora 
criado por três professores para ensinar aos alunos 
como era a vida dos colonos do século XIX nos EUA. 

Embora, de acordo com \cite{poli:2012:video-game-spore}, 
tenha sido demonstrado em alguns casos que o uso de 
jogos em sala de aula pode ser benéfico para o 
envolvimento e desempenho dos alunos, proporcionando 
um aumento de até 3 horas extras por semana de contato 
com o material da classe e até 5\% de notas mais 
altas, a utilização de jogos em sala de aula ainda 
é baixo, quase inexistente. Ainda de acordo com 
\cite{poli:2012:video-game-spore}, as principais causas 
disso são a dificuldade em manter o engajamento de 
alunos, a falta de conteúdo complementar e o orçamento 
limitado.

Considerando que jogar videogames é a quarta atividade 
mais comum entre adolescentes no Brasil, como visto em 
\cite{bndes:2014:mapeamento}, com 35\% das crianças 
entre 9 e 16 anos jogando diariamente, 45\% jogando 
pelo menos uma vez por semana e 19\% jogando uma vez 
ao mês, não se pode negar o potencial deste meio quanto 
ao seu poder de disseminação de conhecimento e cultura, 
de uma maneira similar àquela de peças teatrais, 
literatura e filmes.

Com o amadurecimento e ubiquidade da tecnologia móvel, 
com 75.2\% da população e 49.9\% das crianças entre 10 
e 14 possuindo um celular no Brasil, segundo dados de 
\cite{IBGE:2015:PNAD_TIC_2013}, é necessário um estudo 
aprofundado sobre alternativas para os jogos educacionais 
tradicionais, que combinem tecnologia e conhecimento 
para criar uma solução imersiva e efetiva.

Recentemente, o tópico do desenvolvimento de 
\textit{video games} educativos tem voltado a estar em 
pauta, sobretudo dada a grande revolução tecnológica 
que tem ocorrido no campo de mídias interativas: a 
criação de dispositivos de realidade virtual. 
Iniciativas como a Nearpod VR \cite{nearpod}, que 
permitem a realização de apresentações e excursões 
virtuais, preparam o caminho para o desenvolvimento 
de novas e mais sofisticadas soluções educativa em 
realidade virtual.

É inserido neste panorama que este trabalho se propõe a 
explorar a concepção e desenvolvimento de um \textit{video game} 
que se valha dos potenciais imersivos de tecnologias que 
realidade virtual para criar um ambiente fértil para 
aprendizado lúdico e efetivo.

\section{Objetivos}\label{sec-objetivos}

A partir da realização deste projeto, pretende-se dar os 
primeiros passos rumo à criação de jogos efetivamente 
educacionais, capazes de manter o engajamento e interesse 
de estudantes, ao passo que ensinam sobre tópicos e 
habilidades relevantes para seu desenvolvimento intelectual 
e psicológico.

Utilizando tecnologias de realidade virtual, espera-se 
desenvolver um jogo digital que se mostre recreativo e 
imersivo, e que permita a alunos aprender e se familiarizar 
com os conceitos ensinados por conta própria, através de 
experimentação e exploração, de uma maneira que possa ser 
supervisionada, auxiliada e acompanhada por profissionais 
da área da educação, dentro e fora da sala de aula.

Também pretende-se dar preferência ao uso de tecnologias 
e dispositivos de baixo custo de adoção, de modo a chegar 
a uma solução final  que possa ser implementada em larga 
escala por escolas e instituições educacionais sem impactos 
orçamentários proibitivos.

\section{Justificativa}\label{sec-justificativas}

De acordo com a prova \textit{PISA}, feita para mensurar o 
desempenho acadêmico de alunos de 15 anos, ministrada pela 
\textit{Organisation for Economic Co-operation and Development} 
(OCDE), o Brasil ocupa a 58\textsuperscript{a} posição dentre 
os 65 países estudados quanto a performance de seus estudantes 
em matemática, com uma nota média de 391, contra uma média 
global de 494, como demonstrado em \cite{OECD:2012:pisa-brazil}. 
É importante notar que a nota da prova é normalizada para 
que a média de cada categoria seja 500, com desvio padrão 
de 100 pontos. 

O estudo encontrado em \cite{OECD:2016:low_performing_students}, 
da mesma organização, também revela que dentre os principais 
problemas que levam um aluno a ter uma performance abaixo da média 
destacam-se a falta de engajamento com a matéria em questão, 
o baixo tempo investido em estudo e atividades extra-curriculares, 
e a baixa auto-estima devido a más experiências anteriores 
envolvendo a matéria.

Nestes termos, acreditamos ser importante pensar em maneiras 
de encorajar os alunos a estimular seu raciocínio lógico, 
criando um espaço onde crianças possam se engajar e formar 
conexões positivas com disciplinas na área das ciências 
exatas, preparando-as rumo a um melhor desempenho nestas 
matérias ao médio e longo prazo.

\section{Resultados esperados}\label{sec-resultados-esperados}

Ao término deste projeto espera-se ter um produto final no 
formato de um jogo digital em realidade virtual, que possa 
ser utilizado por educadores para auxiliar no ensino e retenção 
habilidades de raciocínio lógico para crianças em idade escolar; 
bem como um conjunto de especificações técnicas dos 
\textit{softwares} e \textit{hardwares} necessários para sua 
implementação em sala de aula.

Espera-se também ter dados no formato de uma pesquisa 
qualitativa que indiquem uma confirmação ou negação da 
eficácia do nosso produto como ferramenta de ensino capaz 
de educar e entreter crianças.

\section{Trabalhos relacionados}\label{sec-trabalhos-relacionados}

\cite{Tabuti:2010:analise} estudam diferentes interações humano-computador, 
focando em como elas ajudam, ou atrapalham, o aprendizado em ambientes 
digitais. O estudo foi feito com base no cubo de Rubik 
(também conhecido como cubo mágico), tanto em sua forma física 
quando em várias formas digitais.

\cite{SBGames:2015:RVAM} desenvolveram um jogo digital em realidade 
virtual, motivados pela baixa performance de alunos em aulas de matemática. 
O foco do jogo é ajudar o jogador a aprender e reter conceitos apresentados 
em aula, através do uso de ambientes tridimensionais e com o auxílio de 
personagens e mundo visualmente atraentes. O jogo consiste em um mundo 
aberto e uma personagem controlável. O objetivo do jogo é conseguir o 
maior número de moedas o mais rápido possível, enquanto desviando de 
inimigos. Quando o jogador coleta uma moeda, aparece uma tela com uma questão 
de múltipla escolha. Caso o jogador acerte a pergunta, recebe a moeda 
colecionada, caso contrário é teleportado para o início da fase. O jogo pode 
ser dividido em dois estados: um estado de 'jogo', quando o jogador está 
caçando as moedas, e um estado de 'aprendizado', quando o jogador está 
respondendo uma questão.

\cite{Alves:2015:VR_Quimica} também desenvolveram um jogo em realidade 
virtual, mas focando no ensino de química. Os gráficos são muito mais 
realistas do que o exemplo anterior, refletindo o público alvo mais 
maduro. O jogador controla uma personagem que é uma antropomorfização 
de um átomo de hidrogênio, a qual deve explorar um mundo (baseado 
na tabela periódica). Cada área do mapa contém elementos 
diferentes, que interagem com o jogador de maneiras únicas. O objetivo 
do jogo é auxiliar o ensino de reações químicas ao jogador. Através 
da transformação dos elementos em personagens, os autores tinham como 
objetivo ajudar o jogador a entender a química através de metáforas, 
contrapondo o exemplo anterior, onde o jogo é dividido em estados.


\section{Estrutura do trabalho}\label{sec-estrutura-documento}

O \autoref{cap-introducao} é essa introdução, que apresenta os objetivos, justificativas e motivações desse documento, assim como trabalhos relacionados.

O \autoref{cap-teoria} trata da teoria necessária para se 
estabelecer um vocabulário comum e bem definido, explicando 
conceitos e as tecnologias utilizadas na solução apresentada 
por este trabalho.

O \autoref{cap-metodologia} explicita a metodologia utilizada 
neste trabalho para se definir, desenvolver e testar o jogo. 
Explica-se o processo iterativo de design do Ciclo Formal.

O \autoref{cap-desenvolvimento} relata o desenvolvimento do 
jogo, desde sua concepção até sua implementação. Contém 
também especificações importantes, como a definição da 
arquitetura escolhida e as razões para tal.

O \autoref{cap-testes} é referente aos testes realizados com 
jogadores, a fim de validar o jogo. Nele se explica o roteiro 
de testes elaborado, assim como as observações da sessão de testes.

O \autoref{cap-consideracoes-finais} traz as considerações 
finais do trabalho, mencionando os resultados alcançados, 
as dificuldades encontradas e trabalhos futuros.
