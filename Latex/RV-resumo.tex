% resumo em português
\begin{resumo}
\begin{comment}
O resumo deve ser redigido, preferencialmente, na terceira pessoa do singular com verbo na voz ativa, em parágrafo único, e conter no máximo 500 palavras.
\end{comment}

\TODO{Editar resumo.}

Com o massificação do \textit{smartphone}, o número de jogos de celular tem crescido exponencialmente e jogos e aplicativos móveis têm se tornado o principal meio de entretenimento de muitas crianças e adolescentes, que vêm abrindo mão de jogos físicos e brinquedos. Muitos destes jogos físicos, especialmente aqueles voltados às crianças mais jovens, eram projetados com o objetivo de auxiliar no desenvolvimento emocional, intelectual, motor, social e de raciocínio lógico da criança. Já os aplicativos de \textit{smartphone} tendem a ter um \textit{design} que incentiva interações curtas, simplistas e repetitivas, de modo a maximizar o lucro sem fomentar o aprendizado dos jogadores.

Poucos destes jogos são feitos pensando em crianças e, portanto, não são desenvolvidos com o mesmo objetivo dos jogos físicos. Isto revela um quadro preocupante, pois é possível que as crianças nascidas envoltas na tecnologia móvel, ou seja, na era digital, que passam o dia todo brincando com esses dispositivos, tenham um desenvolvimento comprometido se comparado com o das gerações anteriores.

O intuito deste projeto de Trabalho de Conclusão de Curso (TCC) é criar um jogo que, fazendo uso de tecnologia móvel e de realidade virtual, possa providenciar um ambiente imersivo e fértil para a exploração e desenvolvimento de habilidades motoras, lógicas e do raciocínio espacial de crianças, expandindo a fronteira de conhecimento relacionada à interface de jogos digitais aplicados à educação, alinhado ao projeto da doutoranda Lucy Mari Tabuti.

As tecnologias utilizadas serão o Google Cardboard para gerar a realidade virtual e o Leap Motion para a captura de gestos. O jogo será desenvolvido na ferramenta Unity, em virtude das facilidades propiciadas por ela, em especial lidando com ambientes tridimensionais.

 \vspace{\onelineskip}

 \noindent
 \textbf{Palavras-chaves}: Educação, imersão, realidade virtual, raciocínio lógico, jogo.
\end{resumo}
 
% resumo em inglês
\begin{resumo}[Abstract]
 \begin{otherlanguage*}{english}
   The success and adoption rate of video games in education have, up to this day, not proven sufficient to encourage large-scale implementation of gaming as a means to supplement and facilitate learning. This stems from a divergence of the design goals when comparing educational to non-educational digital games, as well as from the practical difficulty of providing each student with the necessary hardware, software and teacher support to ensure the desired results are being achieved from the playing of the games. However, as mobile technology achieves its point of maturity, and affordable, technically viable virtual reality-capable devices start to enter the market en masse, we stand at a critical moment when truly immersive, carefully designed learning experiences can be developed, making use of little more than the smartphones children already carry in their pockets. This paper will detail the conception, implementation and evaluation of such an educational game, highlighting the theoretical basis, design decisions and playtest procedures employed. By the end, it’s expected that the conclusions drawn from the work will allow a stronger understanding of where games stand in respect to providing meaningful learning experiences, and whether mobile and VR technology could drive that capability even further.

    \vspace{\onelineskip}

    \noindent
    \textbf{Key-words}: Education, immersion, virtual reality, logic puzzle, logical reasoning.
  \end{otherlanguage*}
\end{resumo}