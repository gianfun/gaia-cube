% resumo em português
\begin{resumo}
\begin{comment}
O resumo deve ser redigido, preferencialmente, na terceira pessoa do singular com verbo na voz ativa, em parágrafo único, e conter no máximo 500 palavras.
\end{comment}

O sucesso e a taxa de adoção de jogos eletrônicas na educação foi, até os dias atuais, insuficiente para encorajar a implementação em larga escala de experiências lúdicas como formas de facilitar o aprendizado. Isso advém tanto de um divergência nos objetivos de design observados comparando-se jogos educacionais e não-educacionais, como das dificuldades práticas de prover cada estudante com o hardware e software necessários, bem como o auxílio de seus professores que garantam que os resultados desejados sejam obtidos.

Contudo, conforme a tecnologia móvel se aproxima de seu ponto maturidade, e dispositivos de realidade virtual tecnologicamente viáveis e financeiramente acessíveis começam a adentrar o mercado em massa, um momento crítico se aproxima no qual experiências educativas verdadeiramente imersivas e cuidadosamente projetadas possam ser desenvolvidas, valendo-se de pouco mais do que os celulares que as crianças já carregam em seus bolsos.

O intuito deste projeto de Trabalho de Conclusão de Curso (TCC) é criar um jogo que, fazendo uso de tecnologia móvel e de realidade virtual, possa providenciar um ambiente imersivo e fértil para a exploração e desenvolvimento de habilidades motoras, lógicas e do raciocínio espacial de crianças, expandindo a fronteira de conhecimento relacionada a jogos digitais aplicados à educação, alinhado ao projeto da doutoranda Lucy Mari Tabuti.

As tecnologias utilizadas serão o Google Cardboard para gerar a realidade virtual e o Leap Motion para a captura de gestos. O jogo será desenvolvido na ferramenta Unity, em virtude das facilidades propiciadas por ela, em especial lidando com ambientes tridimensionais.

 \vspace{\onelineskip}

 \noindent
 \textbf{Palavras-chaves}: Educação, imersão, realidade virtual, raciocínio lógico, jogo.
\end{resumo}
 
% resumo em inglês
\begin{resumo}[Abstract]
 \begin{otherlanguage*}{english}
   The success and adoption rate of video games in education have, up to this day, not proven sufficient to encourage large-scale implementation of gaming as a means to supplement and facilitate learning. This stems from both a divergence of the design goals when comparing educational to non-educational digital games, as well as from the practical difficulty of providing each student with the necessary hardware, software and support from teachers to ensure the desired results are being achieved from the playing of the games.
   
   However, as mobile technology achieves its point of maturity, and affordable, technically viable virtual reality-capable devices start to enter the market en masse, we stand at a critical moment in which truly immersive, carefully designed learning experiences can be developed, making use of little more than the smartphones children already carry in their pockets.
   
   The purpose of this work is to develop a game that, making use of mobile technology and virtual reality, may provide a fertile and immersive environment for exploration and development of motor, logic and spatial reasoning skills, while expanding the frontiers of educational gaming, aligned to the projects of Professor Lucy Mari Tabuti.
   
   The technologies to be employed are the Google Carboard virtual reality headset and the Leap Motion hand gesture detector. The game will be developed using the Unity engine, given its high level capabilities and suitability to 3D environments.

    \vspace{\onelineskip}

    \noindent
    \textbf{Key-words}: Education, immersion, virtual reality, logic puzzle, logical reasoning.
  \end{otherlanguage*}
\end{resumo}